% !TEX program = xelatex

\documentclass{resume}

\begin{document}
\pagenumbering{gobble} % suppress displaying page number

\name{Jin Fang}

\basicInfo{
  \email{fanjin98@outlook.com}
  \textperiodcentered\
  \phone{(+86) 181-5566-1676}
  \textperiodcentered\
  \scholar[Scholar]{https://scholar.google.com/citations?user=0Hhc6iUAAAAJ}
  \textperiodcentered\
  \homepage[Homepage]{www.fangjin.site}
}

\section{Education}

\datedsubsection{\textbf{University of Science and Technology of China (USTC)}, Anhui, China}{2020.9-present}
\textit{PhD student} in Computer Science (GPA: 3.46/4.00)

\datedsubsection{\textbf{Hunan University (HNU)}, Hunan, China}{2016.9-2020.6}
\textit{B.S.} in Computer Science

\section{Publications}

\datedsubsection{\href{https://ieeexplore.ieee.org/document/10050420}{GRID: Gradient Routing with In-network Aggregation for Distributed Training}}{2023.2}
\role{\textbf{J Fang}, G Zhao, H Xu, C Wu, Z Yu}{IEEE/ACM Transactions on Networking (\textbf{ToN})}

\section{Experience}

\datedsubsection{\href{https://www.fangjin.site/2022/12/02/gradientscheduling/}{\textbf{GOAT}}, Zhijiang Lab open project}{2022.6-2022.9}
\role{Research Intern}{}
GOAT  performs gradient scheduling with collaborative in-network aggregation to efficiently aggregate asynchronously arriving gradients and speed up the distributed training 
\begin{itemize}
  \item Design a knapsack-based randomized rounding algorithm to perform gradient scheduling
  \item Implement GOAT with Pytorch and P4 (TNA) in a testbed containing 8 servers and 3 switches
  \item Reduce the communication overhead of distibuted training tasks by 81.2\%
  % \item Produced a paper (on submission)
\end{itemize}

\datedsubsection{\href{https://github.com/futurewei-cloud/alcor}{\textbf{Alcor}}, Open sourced project}{2020.9-2021.3}
\role{C++ Developer}{}
Alcor leverages the latest SDN and container technologies as well as an advanced distributed system design to support the deployment, configuration, and scale-out of millions of VM and containers.
\begin{itemize}
  \item Developing and end-to-end testing of the virtualization control plane (\href{https://github.com/futurewei-cloud/alcor-control-agent}{ACA})
  \item Add grpc thread for pulsar subscribe infomation (\href{https://github.com/futurewei-cloud/alcor-control-agent/pull/274}{PR \#274})
\end{itemize}

\datedsubsection{\href{https://www.fangjin.site/2021/08/01/RobustVNFPlacement/}{\textbf{Reveal}}, Academic project}{2021.2-2021.6}
Reveal tries to improve the robustness of edge clouds by limiting No. of VNFs each user can access and No. of users each VNF can serve.
\begin{itemize}
  \item Design a two-phase algorithm to solve the problem of VNF placement and request scheduling
  \item Implement Reveal with Python in a testbed containing 6 Nvidia Jetson Tx2s and 20 Raspberry Pis
\end{itemize}

\section{Patents}

\datedsubsection{A gradient scheduling method based on programmable switch under PS architecture}{2021}
\role{G Zhao, \textbf{J Fang}, H Xu, C Wu}{Published: CN114900482A}

\datedsubsection{A VNF placement method in the edge cloud}{2021}
\role{H Xu, \textbf{J Fang}, G Zhao, H Tu, H Wang}{Published: CN113961324A}

% \section{Skills}
% \begin{itemize}[parsep=0.5ex]
%   \item Courses: Algorithm Design
%   \item Development: Pytorch, Mininet
%   \item Programming Languages: C/C++, Python, P4, C\#
% \end{itemize}

\section{Awards}
\begin{itemize}[parsep=0.5ex]
  \item \datedline{Excellent price (25\%) in Intel P4 China Hackthon}{2022}
  \item \datedline{Doctoral first-class academic scholarship}{2022}
  \item \datedline{Master's first-class study scholarship}{2020, 2021}
  % \item \datedline{Teaching Assistant of Advanced Computer Networking}{2022}
\end{itemize}

\end{document}
