% !TEX program = xelatex

\documentclass{resume}

\begin{document}
\pagenumbering{gobble} % suppress displaying page number

\name{Jin Fang}

\basicInfo{
  \email{fanjin98@outlook.com}
  \textperiodcentered\
  \phone{(+86) 181-5566-1676}
  \textperiodcentered\
  \homepage[www.fangjin.site]{www.fangjin.site}
  % \textperiodcentered\
  % \github[Fangjin98]{https://github.com/Fangjin98}
}

\section{Education}

\datedsubsection{\textbf{University of Science and Technology of China (USTC)}}{Anhui, China}
\datedsubsection{\textit{Ph.D.} in Computer Science}{2020.9-2026.6 (expected)}
\begin{itemize}
  \item Research focus on MlSys, Collective Communication, and In-network Computing
  \item Advisors: Prof. Hongli Xu and Prof. Gongming Zhao
\end{itemize}

\datedsubsection{\textbf{Hunan University (HNU)}}{Hunan, China}
\datedsubsection{\textit{B.S.} in Computer Science}{2016.9-2020.6}
\begin{itemize}
  \item Excellent Graduation Thesis of Hunan University
\end{itemize}

\section{Publications}

\begin{enumerate}

  \item S. Zheng, \textbf{J. Fang}, X. Zheng, Q. Hou, W. Bao, N. Zheng, Z. Jiang, D. Wang, J. Ye, H. Lin, L. Chang, X. Liu, \textit{TileLink: Generating Efficient Compute-Communication Overlapping Kernels using Tile-Centric Primitives}, (\textbf{MLSys'25})
  \item \textbf{J. Fang}, G. Zhao, H. Xu, L. Luo, Z. Yao, A. Xie, \textit{Non-Idle Machine-Aware Worker Placement for Efficient Distributed Training in GPU Clusters}, IEEE International Conference on Network Protocols (\textbf{ICNP'24})
  \item \textbf{J. Fang}, G. Zhao, H. Xu, Z. Yu, B. Shen, L. Xie, \textit{Accelerating Distributed Training with Collaborative In-network Aggregation}, IEEE/ACM Transactions on Networking (\textbf{ToN'24})
  \item \textbf{J. Fang}, G. Zhao, H. Xu, Z. Yu, B. Shen, L. Xie, \textit{GOAT: Gradient Scheduling with Collaborative In-Network Aggregation for Distributed Training}, IEEE/ACM International Symposium on Quality of Service (\textbf{IWQoS'23})
  \item \textbf{J. Fang}, G. Zhao, H. Xu, C. Wu, Z. Yu, \textit{GRID: Gradient Routing with In-network Aggregation for Distributed Training}, IEEE/ACM Transactions on Networking (\textbf{ToN'23})
  \item \textbf{J. Fang}, G. Zhao, H. Xu, H. Tu, H. Wang, \textit{Reveal: Robustness-Aware VNF Placement and Request Scheduling in Edge Clouds}, Computer Networks (\textbf{ComNet'23})
  % \item \textbf{J. Fang}, G. Zhao, H. Xu, Z. Yu, J. Jiang, F. Zeng, Injecting Failure for Success: Towards General,
  % Flexible and Efficient Network Fault Injection, USENIX ATC, 2024 (\textit{In submission})
  \item J. Liu, Y. Zhai, G. Zhao, H. Xu, \textbf{J. Fang}, Z. Zeng, Y. Zhu, InArt: In-Network Aggregation with Route
  Selection for Accelerating Distributed Training, International World Wide Web Conference (\textbf{WWW'24})
\end{enumerate}

\section{Experience}

\datedsubsection{\textbf{Communication Collective Library for Sequence Parallelism LLM Job}}{Bytedance Seed-Foundation-mlsys, Beijing, China}
\datedsubsection{\textit{Main Developer}}{2024.11-present}
\begin{itemize}
  \item Implement AllGather and AlltoAll operations based on RDMA verbs programming
  \item Design collective communication algorithms for cross-pcie and cross-node scenarios 
  \item Optimize bandwidth utilization under resource-constraint machine architectures 
  \item Reduce No. of NICs per machine from 8 to 2 (by 75\%), while achieve the same bandwidth utilization
  \item Evaluate performance of different machine architecture and analyze communication bottleneck for different SP setup (SP-Ulysses and SP-Ring)
\end{itemize}

\datedsubsection{\textbf{Communication-Computation Fused GPU Kernel Generation}}{Bytedance Seed-Foundation-mlsys, Beijing, China}
\datedsubsection{\textit{Research Intern}}{2024.6-2024.12}
\begin{itemize}
  \item Implement collective communication operations (e.g., AllGather) based on Triton and NVSHMEM
  \item Design and implement communication-computation fused operations (e.g., AllGather+GEMM), exploring overlaps between GEMM and collective communications
  \item Achieve near-optimized bandwidth utilization on A100*8 NVlink machines
  \item Implement TP-fused and SP-fused kernels for both dense and MoE LLMs (Integrated into 6 popular LLMs)
  \item Implement and optimize cross-node communication computation fusion kernels
  \item Achieve end-to-end speed up by $3\times$ and $1.5\times$ compared with Pytorch and vLLM
\end{itemize}

\datedsubsection{\textbf{Optimizing Worker Placement for Distributed Training in OCS Network}}{Huawei 2012 Lab, Hefei, China}
\datedsubsection{\textit{Research Intern}}{2023.12-2024.5}
\begin{itemize}
  \item Investigate existing large model task deployment and resource scheduling works
  \item Investigate existing gradient compression optimization for sparse model training
  \item Model physical and logical communication patterns of different collective communication algorithms, analyze the impact of communication topology on task training time
  \item Design a task placement algorithm to optimize the cross-rack traffic in the optical circuit switch network
\end{itemize}

\datedsubsection{\textbf{Simulating network faults with programmable dataplane}}{Suzhou, China}
\datedsubsection{\textit{Main Developer}}{2022.12-2023.9}
\begin{itemize}
  \item Build a user-friendly, multi-backend fault injection system in programmable dataplane
  \item Design a parser generation algorithm to handle flow dependency and load the table entries
  \item Formulate the fault injection point selection problem
  \item Implement several network faults with P4 in TNA and PSA architectures
\end{itemize}

\datedsubsection{\textbf{Accelerating distributed training with programmable switches}}{Zhijiang Lab, Hangzhou, China}
\datedsubsection{\textit{Research Intern}}{2022.6-2022.9}
\begin{itemize}
  \item Improve the in-network aggregation throughput by mitigating the influence of asychronous arrived packets
  \item Design a knapsack-based randomized rounding algorithm for gradient scheduling
  \item Implement a distributed training prototype with Pytorch 
  \item Implement the in-network aggregation logic in Tofino with P4
  \item Reduce the communication overhead of distibuted training tasks by 81.2\%
\end{itemize}

\datedsubsection{\textbf{Developing and testing Alcor, a cloud native SDN platform}}{Futurewei, \textit{Remotely}}
\datedsubsection{\textit{Developer}}{2021.6-2021.9}
\begin{itemize}
  \item Write an automatic building script for large scale deployment with bash
  \item Write an end-to-end test of the virtualization control plane (\href{https://github.com/futurewei-cloud/alcor-control-agent}{ACA}) with C++
  \item Develop grpc thread for pulsar subscribe information (\href{https://github.com/futurewei-cloud/alcor-control-agent/pull/274}{PR \#274}) with C++
\end{itemize}

% \datedsubsection{\textbf{Robust-awareness VNF placement in the edge cloud}}{Hefei, China}
% \datedsubsection{\textit{Main Developer}}{2021.2-2021.6}
% \begin{itemize}
%   \item Improve the robustness of edge clouds by limiting the influence of malicious users and failed VNFs
%   \item Design a two-phase algorithm to solve the problem of VNF placement and request scheduling
%   \item Implement a prototype containing 6 Nvidia Jetson Tx2s and 20 Raspberry Pis with Python
%   \item Improve the network throughput by $57\%$ under exisitence the malicious user
% \end{itemize}

\datedsubsection{\textbf{Implement a LSTM model based on high-level synthesis}}{Hunan, China}
\datedsubsection{\textit{Main Developer}}{2019.6-2020.1}
\begin{itemize}
  \item Train a LSTM model based on Keras to predict the steam pressure in nuclear power plant reactor
  \item Implement the trained LSTM model with C++ and deploy it into a Pynq-Z2 board
  \item Reduce the inference time by 90x compared with software implementation
  \item \textit{Win the award of Excellent Graduation Thesis of Hunan University}
\end{itemize}

% \section{Patents}

% \begin{enumerate}
%   \item G. Zhao, \textbf{J. Fang}, H. Xu, C. Wu, \textit{A gradient scheduling method based on programmable switch under PS architecture}, CN114900482B
%   \item H. Xu, \textbf{J. Fang}, G. Zhao, H. Tu, H. Wang, \textit{A VNF placement method in the edge cloud}, CN113961324A
% \end{enumerate}

% \section{Courses}

% \begin{itemize}[parsep=0.5ex]
%   \item \datedline{COMP6002P, Combinatorial Mathematics}{89/100}
%   \item \datedline{COMP6201P, Parallel Programming}{86/100}
%   \item \datedline{COMP7102P, Advanced Algorithm Design and Analysis}{91/100}
% \end{itemize}

\section{Awards}

\begin{itemize}[parsep=0.5ex]
  \item \datedline{Guorui scholarship}{2023}
  \item \datedline{Excellent price (25\%) in Intel P4 China Hackthon}{2022}
  \item \datedline{Doctoral first-class academic scholarship}{2022, 2023}
  \item \datedline{Master's first-class study scholarship}{2020, 2021}
\end{itemize}

\section{Skills}

\begin{itemize}[parsep=0.5ex]
  \item Programming Language: C/C++, Python, P4, C\#, Swift
  \item Developing Framework: Pytorch, p4c, eBPF, Mininet
\end{itemize}

% \section{Services}

% \begin{itemize}[parsep=0.5ex]
%   \item External Reviewer: IEEE JSAC, IEEE TNET, COMNET
%   \item Teaching Assistant: COMP6103P Advanced Computer Networking
% \end{itemize}

\end{document}
