% !TEX program = xelatex

\documentclass{resume}

\begin{document}
\pagenumbering{gobble} % suppress displaying page number

\name{Jin Fang}

\basicInfo{
  \email{fanjin98@outlook.com}
  \textperiodcentered\
  \phone{(+86) 181-5566-1676}
  \textperiodcentered\
  \github[Fangjin98]{https://github.com/Fangjin98}
  \textperiodcentered\
  \homepage[www.fangjin.site]{www.fangjin.site}
}

\section{Education}

\datedsubsection{\textbf{University of Science and Technology of China (USTC)}}{Anhui, China}
\datedsubsection{\textit{PhD student} in Computer Science (GPA: 3.46/4.00)}{2020.9-present}

\datedsubsection{\textbf{Hunan University (HNU)}}{Hunan, China}
\datedsubsection{\textit{B.S.} in Computer Science}{2016.9-2020.6}

\section{Publications}

\begin{enumerate}
  \item \textbf{J. Fang}, G. Zhao, H. Xu, Z. Yu, B. Shen, X. Li, \textit{GOAT: Gradient Scheduling with Collaborative In-Network Aggregation for Distributed Training}, IEEE/ACM International Symposium on Quality of Service (\textbf{IWQoS'23})
  \item \textbf{J. Fang}, G. Zhao, H. Xu, C. Wu, Z. Yu, \textit{GRID: Gradient Routing with In-network Aggregation for Distributed Training}, IEEE/ACM Transactions on Networking (\textbf{ToN'23})
\end{enumerate}

\section{Experience}

\datedsubsection{\textbf{Speeding up distributed training with programmable switches}}{Zhijiang Lab, Hangzhou, China}
\datedsubsection{\textit{Research Intern}}{2022.6-2022.9}
\begin{itemize}
  \item Improved the in-network aggregation throughput by mitigating the influence of asychronous arrived packets
  \item Designed a knapsack-based randomized rounding algorithm for gradient scheduling
  \item Implemented a distributed training prototype with Pytorch 
  \item Implemented the in-network aggregation logic in Tofino with P4 %TODO: how to solve resource isolation.
  \item Reduced the communication overhead of distibuted training tasks by 81.2\%
\end{itemize}

\datedsubsection{\textbf{Robust-awareness VNF placement in the edge cloud}}{Hefei, China}
\datedsubsection{\textit{Research Assistant}}{2021.2-2021.6}
\begin{itemize}
  \item Improved the robustness of edge clouds by limiting the influence of malicious users and failed VNFs.
  \item Designed a two-phase algorithm to solve the problem of virtual network functions placement and request scheduling
  \item Implemented a prototype containing 6 Nvidia Jetson Tx2s and 20 Raspberry Pis with Python
  \item Improved the network throughput by $57\%$ under exisitence the malicious user.
\end{itemize}

\datedsubsection{\textbf{Developing and testing Alcor, a cloud native SDN platform}}{Hefei, China}
\datedsubsection{\textit{Developer}}{2020.9-2021.3}
\begin{itemize}
  \item Wrote an automatic building script for large scale deployment with bash.
  \item Wrote an end-to-end test of the virtualization control plane (\href{https://github.com/futurewei-cloud/alcor-control-agent}{ACA}) with C++
  \item Added grpc thread for pulsar subscribe information (\href{https://github.com/futurewei-cloud/alcor-control-agent/pull/274}{PR \#274}) with C++
\end{itemize}

\section{Patents}

\begin{enumerate}
  \item G. Zhao, \textbf{J. Fang}, H. Xu, C. Wu, \textit{A gradient scheduling method based on programmable switch under PS architecture}, Published: CN114900482A
  \item H. Xu, \textbf{J. Fang}, G. Zhao, H. Tu, H. Wang, \textit{A VNF placement method in the edge cloud}, \\ Published: CN113961324A
\end{enumerate}

\section{Awards}

\begin{itemize}[parsep=0.5ex]
  \item \datedline{Excellent price (25\%) in Intel P4 China Hackthon}{2022}
  \item \datedline{Doctoral first-class academic scholarship}{2022}
  \item \datedline{Master's first-class study scholarship}{2020, 2021}
\end{itemize}
% \item \datedline{Teaching Assistant of Advanced Computer Networking}{2022}

% \section{Skills}
% \begin{itemize}[parsep=0.5ex]
%   \item Courses: Algorithm Design
%   \item Development: Pytorch, Mininet
%   \item Programming Languages: C/C++, Python, P4, C\#
% \end{itemize}

\end{document}
